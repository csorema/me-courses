\documentclass[a4paper,12pt]{article}

% Set margins
\usepackage[hmargin=2.5cm, vmargin=2cm]{geometry}

\frenchspacing

% Language packages
\usepackage[utf8]{inputenc}
\usepackage[T1]{fontenc}
\usepackage[magyar]{babel}

% AMS
\usepackage{amssymb,amsmath}

% Graphic packages
\usepackage{graphicx}

% Colors
\usepackage{color}
\usepackage[usenames,dvipsnames]{xcolor}

% Enumeration
\usepackage{enumitem}

% Links
\usepackage{hyperref}

\pagestyle{empty}

\begin{document}

\begin{center}
	{\Large ÜTEMTERV -- \texttt{GEAGT131-B, GEAGT131-B2}}

	\bigskip

	{\huge \textbf{Számítógépi grafika}}
	
	\medskip
	
	{\large Gazdaságinformatikus, Mérnökinformatikus, Programtervező informatikus BSc szakok,
	
	nappali tagozat, tavaszi félév}

	\medskip
	
	\textit{Előfeltétel: Lineáris algebra, Objektum orientált programozás}
\end{center}

\vskip 5mm

\noindent \textbf{1. hét}:
Koordinátarendszerek, vektorok, normál vektorok, transzformációk, mátrix verem, színterek, színmélység, pont-egyenes, pont-sík távolsága

\bigskip

\noindent \textbf{2. hét}:
Grafikus input és output eszközök, raszteres és vektorgrafikus megjelenítés

\bigskip

\noindent \textbf{3. hét}:
Rajzoló algoritmusok: szakaszok rajzolása, élsimítás, háromszög kitöltés, ellipszis rajzolása, képelemek vágása, floodfill algoritmus, szoftveres eszközök, GUI toolkit-ek, esemény- és programvezérelt programozás, interaktív grafikus felületek kialakítása, az SDL2 függvénykönyvtár

\bigskip

\noindent \textbf{4. hét}:
Térbeli objektumok ábrázolása síkon: perspektivikus és orthogonális vetítés, a kamera működése és beállítása

\bigskip

\noindent \textbf{5. hét}:
Láthatóság, takarási probléma, Z-buffer, triviális lapeldobás, félig átlátszó felületek megjelenítése

\bigskip

\noindent \textbf{6. hét}:
Árnyalás, fények, anyagjellemzők, árnyaló (\textit{shader}) programozás

\bigskip

\noindent \textbf{7. hét}:
Textúrázás, képformátumok, mip-maps, textúra piramisok

\bigskip

\noindent \textbf{8. hét}:
Animáció, csontozás, modellformátumok (pl.: WaveFront OBJ)

\bigskip

\noindent \textbf{9. hét}:
Fotorealisztikus képszintézis: sugárkövetés, globális illumináció

\bigskip

\noindent \textbf{10. hét}:
Grafikus szabványok, rendszerek: modern OpenGL, DirectX, Vulkan, WebGL

\bigskip

\noindent \textbf{11. hét}:
Speciális megjelenítési módok: árnyék, víz, tűz, stencil buffer, tükrök

\bigskip

\noindent \textbf{12. hét}:
Szövegek megjelenítése: fontok, raszterizálás, kerning, GUI elemek

\bigskip

\noindent \textbf{13. hét}:
Ütközésvizsgálat, LOD, environment mapping, bump mapping, displacement mapping

\bigskip

\noindent \textbf{14. hét}:
Aktuális kutatások, eredmények a számítógéi grafikában: light paths, Gaussian splatting, Unreal Engine, Unity Engine

\newpage

\noindent \textbf{Az aláírás megszerzésének feltétele}

\noindent A gyakorlatokon való aktív részvétel. Egyéni feladat elkészítése és bemutatása.

\vskip 1cm

\noindent \textbf{A vizsga}

\noindent A vizsga írásbeli lesz, mely egyaránt tartalmaz elméleti és gyakorlati kérdéseket is. A vizsgához az 1-8. hét témái tartoznak.

\vskip 1cm

\noindent \textbf{Ponthatárok}

\noindent A dolgozatokon maximálisan 12 pontot lehet szerezni. Az érdemjegyekre az alábbi ponthatárok vonatkoznak.

\begin{center}
\begin{tabular}{|c|c|}
pont & érdemjegy \\
\hline
0-5 & 1 \\
6 & 2 \\
7-8 & 3 \\
9-10 & 4 \\
11-12 & 5 \\
\hline
\end{tabular}
\end{center}

\vskip 18mm

\hskip 10cm Piller Imre

\hskip 7cm Alkalmazott Matematikai Intézeti Tanszék

\vskip 5mm

\noindent Miskolc, 2024. február 6.

\end{document}
